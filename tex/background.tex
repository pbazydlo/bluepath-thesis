
\chapter{Podstawy teorytyczne}

W tym rozdziale przedstawione zostaną terminy wykorzystywane w dalszej
części pracy oraz zestawienie rozwiązań podobnych tematyką do tej
pracy.


\section{Przetwarzanie równoległe}

\label{sec:background-rownolegle}Jednym z modeli przetwarzania danych
jest przetwarzanie równoległe. Polega ono na przetwarzaniu zbioru
danych jednocześnie na wielu procesorach i rdzeniach. W ten sposób
fragmenty przetwarzania, które nie są od siebie zależne mogą zostać
wykonane w tym samym czasie -- pozwala to skrócić czas przetwarzania.

Jednym ze sposobów prowadzenia przetwarzania równoległego jest wykorzystanie
wątków. Wątki, czasami nazywane lekkimi procesami, wykonują kod niezależnie
od siebie, ale działają we wspólnej przestrzeni adresowej, co pozwala
współdzielić dane, ale wymaga zastosowania mechanizmów synchronizacji.
Podstawowym mechanizmem synchronizacji są zamki. Służą one do realizacji
sekcji krytycznych -- fragmentów kodu, które mogą być wykonywane w
danym momenie tylko przez jeden wątek. 


\section{Przetwarzanie rozproszone}

\label{sec:background-Przetwarzanie-rozproszone} Przetwarzanie rozproszone~\cite{ArpaciDusseau14-Book}
jest szeroko stosowanym podejściem w przypadku przetwarzania dużej
ilości danych, konieczności obsługi wielu zleceń użytkowników w jednostce
czasu czy zapewnienia niezawodności systemu. Problemy, z jakimi zmagają
się programiści aplikacji rozproszonych dotyczą m. in. obsługi awarii
(wraz z rosnącą liczbą elementów systemu rośnie prawdopodobieństwo
awarii jednego z~nich), bezpieczeństwa (zapewnienia poufności, integralności
i autentyczności komunikatów) i komunikacji. Trzeba brać pod uwagę
możliwość utraty pakietu, niedostarczenia potwierdzenia odbioru, maksymalizować
wydajność i przepustowość poprzez minimalizowanie liczby komunikatów
i opóźnień komunikacyjnych. Wykorzystywane w praktyce protokoły sieciowe
rozwiązują także wiele innych problemów, jak np.~zatory w sieci czy
zaległe żądania (ang.~\dcsemph{outstanding requests}). W celu przyspieszenia
tworzenia systemów rozproszonych zaproponowano wprowadzenie abstrakcji
realizujących niezawodną komunikację sieciową i wymianę danych bez
bezpośredniego udziału programisty. 


\subsection{DSM}

\label{sub:background-DSM}Model rozproszonej pamięci współdzielonej
(ang.~\dcsemph{distributed shared memory}, DSM)~\cite{DSM} implementowany
jest na poziomie systemu operacyjnego. Zakłada on udostępnienie wspólnej
przestrzeni adresowej dla wszystkich komputerów biorących udział w
przetwarzaniu. Aplikacje odwołują się do pamięci w tradycyjny sposób.
W przypadku wystąpienia błędu braku strony system operacyjny zajmuje
się sprowadzeniem jej do lokalnej pamięci. Mechanizm DSM nie jest
obecnie wykorzystywany przy konstruowaniu systemów rozproszonych.


\subsection{RPC}

\label{def:background-RPC}Zdalne wywoływanie procedur (ang.~\dcsemph{remote procedure call},
RPC) \cite{RPC} jest realizowane na poziomie języka programowania.
Jego celem jest sprawić, by wywołanie metody na zdalnym komputerze
było dla programisty tak proste, jak wywołanie lokalnej metody. Zadaniem
serwera jest zdefiniowanie listy metod, które udostępnia i które mogą
być wywoływane przez klienta. Zdalne wywoływanie procedur składa się
z biblioteki odpowiedzialnej za realizację komunikacji oraz korzystających
z niej, generowanych automatycznie, namiastki klienta (ang.~\dcsemph{client stub})
i namiastki serwera (ang.~\dcsemph{server stub}). Namiastka klienta
odpowiada za:
\begin{itemize}
\item utworzenie bufora na wiadomość, 
\item wypełnienie go danymi -- umieszczenie w nim jakiegoś rodzaju wskaźnika/uchwytu
do metody oraz parametry wywołania -- w procesie serializacji (ang.~\dcsemph{serialization},
\dcsemph{marshalling}), 
\item wysłanie wiadomości przy pomocy biblioteki realizującej proces komunikacji,
\item dekodowanie otrzymanego rezultatu -- proces ten jest nazywany deserializacją
(ang.~\dcsemph{deserialization}, \dcsemph{unmarshalling}).
\end{itemize}
Namiastka serwera jest symetryczna względem namiastki klienta i do
jej zadań należą:
\begin{itemize}
\item dekodowanie otrzymanego zlecenia,
\item faktyczne wywołanie wskazanej metody,
\item przygotowanie wiadomości zawierającej wynik,
\item zlecenie wysłania odpowiedzi do klienta.
\end{itemize}
Głównym problemem w modelu RPC są: serializacja złożonych struktur
danych, obsługa wskaźników oraz obsługa współbieżnych żądań przez
serwer.


\subsection{Problem detekcji zakończenia}

\label{sub:background-Problem-detekcji-zakonczenia}Istotnym problemem
jest detekcja zakończenia przetwarzania w systemie rozproszonym. Konstrukcja
obrazu globalnego stanu systemu w celu oceny stanu poszczególnych
węzłów jest problemem trudnym. Istnieje szereg algorytmów, które go
adresują. Wymagają one wprowadzenia pewnych założeń co do właściwości
systemu, jak np. kanały FIFO w algorytmie detekcji zakończenia dla
systemów asynchronicznych \cite{Misra83} czy drzewiasta topologia
przetwarzania w algorytmie Dijkstry-Scholtena dla modelu dyfuzyjnego
\cite{Dijkstra-Diffusing}.


\section{Istniejące rozwiązania}

\label{sec:background-Istniej=000105ce-rozwi=000105zania}Poniżej
przedstawiono wybrane środowiska umożliwiające tworzenie aplikacji
do rozproszonych obliczeń. W tabeli \ref{tab:background-Podsumowanie-istniejacych-rozwiazan}
znajduje się ich porównanie.


\subsection{PVM}

PVM (ang.~\dcsemph{Parallel Virtual Machine}) \cite{PVM} powstał
w 1989 r. w Oak Ridge National Laboratory, a jego dalszy rozwój odbywał
się na University of Tennessee. W 1993 r. została wydana wersja 3.0.
Jest to zintegrowany zestaw bibliotek i narzędzi, które służą do projektowania
aplikacji rozproszonych działąjących w środowisku połączonych siecią
komputerów heterogenicznych. Główne założenia systemu obejmowały:
\begin{itemize}
\item używanie puli maszyn określonej przez użytkownika do realizacji przetwarzania
oraz możliwość dodawania i usuwania komputerów podczas pracy,
\item brak pełnego abstrahowania od typów maszyn, tj. aplikacje, które wykorzystywały
specyficzne własności określonych maszyn mogły zostać im przypisane,
\item model przetwarzania oparty o zadania, które można utożsamiać z procesami
w systemie Unix oraz jawne przesyłanie komunikatów pomiędzy zadaniami,
\item wsparcie dla heterogenicznych środowisk -- różnych typów komputerów,
sieci i~aplikacji, a także obsługa sytuacji, w których różne maszyny
korzystają z~różnych reprezentacji danych.
\end{itemize}

\subsection{MPI}

MPI (ang.~\dcsemph{Message-Passing Intefrace}) \cite{MPI-Standard-Version-3,Using-MPI-Portable-Parallel-Programming-with-the-Message-Passing-Interface}
jest standardem opisującym interfejs biblioteki wspomagającej tworzenie
systemów opartych na przesyłaniu wiadomości (ang.~\dcsemph{message-passing systems}).
Standard MPI skupia się na modelu programowania opartym o przesyłanie
wiadomości. Definiuje również współbieżne operacje wejścia/wyjścia,
dynamiczne tworzenie procesów oraz operacje zbiorowe -- np. synchronizacja
procesów przy pomocy barier (ang. \dcsemph{barrier synchronization}).
Jednymi z głównych założeń jakie postawili sobie twórcy standardu
są:
\begin{itemize}
\item zapewnienie wydajnej i wiarygodnej komunikacji,
\item umożliwienie stworzenia implementacji działających w środowisku heterogenicznym,
\item wygodny w wykorzystaniu interfejs dla języka C oraz Fortran.
\end{itemize}

\subsection{Dryad i DryadLINQ}

System Dryad \cite{Microsoft-Research-Dryad} został stworzony przez
zespół naukowców z Microsoft Research. Jego celem było zapewnienie
niezawodnego środowiska do obliczeń rozproszonych na dużych zbiorach
danych. System ten miał pozwalać programiście pisać programy wykonywane
w klastrze bez posiadania umiejętności programowania równoległego
bądź rozproszonego.

Przetwarzanie było modelowane jako skierowany graf acykliczny, w którym
wierzchołki reprezentowały sekwencyjne programy, a krawędzie przepływ
danych jednokierunkowymi kanałami. System Dryad na podstawie tak zamodelowanego
przetwarzania tworzył graf przetwarzania, wykonywał go i w razie potrzeby
modyfikował.

Programy w wierzchołkach były zapisywane przy pomocy języka SCOPE
\linebreak{}
(ang.~\dcsemph{Structured Computations Optimized for Parallel Execution})
\cite{SCOPE}. Język ten posiadał składnię podobną do języka SQL (and.
\dcsemph{Structured Query Language}) \cite{SQL}.

W celu uproszczenia przetwarzania na bazie systemu Dryad powstało
środowisko DryadLINQ \cite{DryadLINQ-MSR}. Środowisko to dostarczało
implementację LINQ (ang.~\dcsemph{Language Integrated Query}), która
wykorzystywała środowisko Dryad do rozproszonego przetwarzania kolekcji.
Jednym z ograniczeń wprowadzonych przez DryadLINQ w stosunku do LINQ
było założenie, że funkcje przetwarzające obiekty kolekcji nie będą
modyfikowały zmiennych zdefiniowanych poza nimi -- w przypadku wykonania
takiej operacji twórcy nie definiowali zachowania systemu.

W roku 2011 Microsoft zaprzestał rozwijania Dryad w ramach \dcsemph{High Performance Comupting Pack}
\cite{DryadLINQ-discarded}, skupiając się na dostosowaniu Apache
Hadoop do pracy pod kontrolą Windows Server i Windows Azure.


\subsection{Hadoop}

Przetwarzanie w modelu mapowania i redukcji -- MapReduce \cite{Google-MapReduce}
-- zostało pierwotnie zaproponowane przez firmę Google. Hadoop to
jego implementacja wydana na otwartej licencji. Środowisko korzysta
z \dcsemph{Hadoop Distributed File System} (HDFS)~\cite{HDFS-Architecture-Guide}
-- rozproszonego systemu plików o wysokiej odporności na awarie węzłów.
Udostępnia on dane w sposób strumieniowy oraz jest dostosowany do
obsługi dużych zbiorów danych (tj. rozmiaru mierzonego w terabajtach)
i dużych plików. Twórcy środowiska wyszli z założenia, że ,,zmiana
miejsca obliczeń jest mniej kosztowna niż przesłanie danych'', w związku
z tym HDFS udostępnia aplikacjom interfejs, który pozwala zmienić
miejsce wykonania tak, aby odbyło się tam, gdzie fizycznie znajdują
się dane. 


\subsection{Spark}

Środowisko obliczeniowe Spark \cite{Spark-Berkeley} powstało w wyniku
prac prowadzonych na University of California w Berkeley. Od 2013
r. jako Apache Spark jest rozwijane przez Apache Software Foundation.
Środowisko to może być użyte do przetwarzania w~modelu nieograniczonym
do dwóch faz jak w MapReduce, co upodabnia je bardziej do Dryad. Zastosowane
w środowisku optymalizacje pozwoliły znacząco poprawić czas opóźnienia
rozpoczęcia przetwarzania zadania w klastrze składającym się z tysięcy
rdzeni \cite{Spark-Berkeley-Performance}. Warto wspomnieć w tym miejscu
o ciekawej abstrakcji pamięci udostępnianej przez środowisko Spark
o nazwie \dcsemph{Resilient Distributed Dataset} (RDD) zapewniającej
odtwarzanie danych utraconych w wyniku awarii. 


\subsection{Erlang}

Język programowania Erlang \cite{Erlang-Armstrong:2010:ERL:1810891.1810910}
został zaprojektowany do tworzenia odpornych na awarie systemów rozproszonych.
Jego historia sięga roku 1985, kiedy to w firmie Ericsson postanowiono
,,zrobić coś ze sposobem, w jaki tworzą aplikacje''. W roku 2000 Erlang
został upubliczniony na otwartej licencji. Podstawowe jego cechy to:
\begin{itemize}
\item izolacja procesów, brak współdzielonych struktur danych w pamięci
operacyjnej, zamków, semaforów,
\item procesy komunikują się przesyłając asynchroniczne wiadomości, które
zawierają faktyczne dane, a nie referencje na zdalne obiekty,
\item zdolność do wykrycia awarii i replikacja -- procesy otrzymują sygnały
w przypadku awarii obserwowanych procesów i muszą posiadać tyle danych,
by przejąć zadania straconego węzła i kontynuować przetwarzanie,
\item system udostępnia metodę określania przyczyn awarii procesu.
\end{itemize}
Z uwagi na mały narzut pamięciowy procesów (nazywanych także \dcsemph{aktorami})
programista może tworzyć ich bardzo wiele. 


\subsection{Project Orleans}

W modelu programowania Orleans \cite{Orleans-export:210931} zaproponowano
wprowadzenie abstrakcji \dcsemph{wirtualnych aktorów}. W porównaniu
do języka Erlang, środowisko wykonwacze działa tu na wyższym poziomie
abstrakcji, uwalniając programistę od problemów takich jak obsługa
awarii, odtwarzanie aktorów i zarządzanie rozproszonymi zasobami --
w szczególności rozłożeniem aktorów na poszczególne węzły. W systemie
Orleans przyjęto następujące założenia:
\begin{itemize}
\item aktorzy istnieją permanentnie (wirtualnie), programista nie tworzy
ani nie niszczy samodzielnie reprezentujących ich obiektów,
\item instancje aktorów (tzw. \dcsemph{aktywacje}) są tworzone automatycznie
w momencie wysłania żądania do aktora, który nie istnieje lub gdy
aktor jest typu \dcsemph{stateless worker}, tj. nie jest wymagane,
aby istaniała tylko jedna instancja aktora danego typu (w przeciwieństwie
do \dcsemph{single activation}),
\item przezroczystość położenia -- instancja aktora może istnieć na różnych
maszynach w różnych chwilach, może też nie istnieć w ogóle fizycznie,
aplikacja nie zna lokalizacji aktora,
\item automatyczne skalowanie -- niezależne instancje aktorów typu \dcsemph{stateless worker}
mogą być uruchamiane w wielu egzemplarzach.
\end{itemize}

\subsection{Podsumowanie}

Nie ulega wątpliwości, że~projektując nowe systemy warto trzymać
się rozwiązań, które są aktywnie rozwijane i~wspierane. Istotnym
czynnikiem decydującym o wyborze lub odrzuceniu środowiska może być
język programowania. Pomimo tego, że język jest tylko narzędziem i~dla
programisty, który poznał ich kilka, rozpoznanie kolejnego i napisanie
w nim aplikacji zazwyczaj nie stanowi problemu, okazuje się, że część
osób posiada preferowany język programowania. Podczas pisania w nim
czuje się komfortowo, wie jakich problemów może się spodziewać i jak
je rozwiązać. W związku z tym w tabeli \ref{tab:background-Podsumowanie-istniejacych-rozwiazan}
zestawiającej omówione środowiska obok modelu programowania uwzględniono
także języki programowania wspierane przez te środowiska oraz wskazano
rok opublikowania pierwszego i najnowszego wydania.

\begin{sidewaystable}
\centering{}\protect\caption{\label{tab:background-Podsumowanie-istniejacych-rozwiazan}Porównanie
wybranych środowisk przetwarzania rozproszonego}
\begin{tabular}{|c|c|c|c|}
\hline 
Środowisko & Język programowania & Model programowania & Pierwsze\textsuperscript{a}/najnowsze\tabularnewline
 &  &  & wydanie\tabularnewline
\hline 
\hline 
PVM & C/C++/Fortran & przesyłanie komunikatów & 1989 / 2009\tabularnewline
\hline 
MPI & C/C++/Fortran & przesyłanie komunikatów & 1994 / 2014\textsuperscript{b}\tabularnewline
\hline 
Dryad & SCOPE / .NET (DryadLINQ) & przepływ danych w DAG & 2009 / 2011\tabularnewline
\hline 
Hadoop & Java/Scala, Python/C++ \cite{Hadoop-Python} & mapowanie i reducja & 2009 / 2014\tabularnewline
\hline 
Spark & Java/Scala/Python & niesprecyzowany & 2014 / 2014\tabularnewline
\hline 
Erlang & Erlang & przesyłanie komunikatów & 2000 / 2014\tabularnewline
\hline 
Orleans & .NET (C\#/F\#/VB) & przesyłanie komunikatów & 2014 / 2014\tabularnewline
\hline 
\multicolumn{4}{l}{\textsuperscript{a}Pierwsze publiczne stabilne wydanie}\tabularnewline
\multicolumn{4}{l}{\textsuperscript{b}Implementacja Open MPI}\tabularnewline
\end{tabular}
\end{sidewaystable}



\section{Definicje}

Poniżej zdefiniowano pojawiające się w pracy pojęcia oraz wyjaśniono
istotne terminy technologiczne.


\subsection{Nazewnictwo}

W tym miejscu należy zdefiniować pojęcia \dcsemph{biblioteka Bluepath},
\dcsemph{użytkownik} oraz \dcsemph{kod użytkownika}, które mają
specjalne znaczenie.
\begin{defn}
~

\noindent \dcsstrong{Biblioteka Bluepath} (w skrócie: \dcsstrong{biblioteka})
jest przedmiotem niniejszej pracy. To zestaw klas, które wspomagają
twórców aplikacji rozproszonych.
\end{defn}

\begin{defn}
~

\noindent \label{def:podstawy-teorytyczne-uzytkownik}Mianem \dcsstrong{użytkownika}
określany jest programista korzystający z funkcji udostępnianych przez
bibliotekę Bluepath podczas tworzenia aplikacji.
\end{defn}

\begin{defn}
~

\noindent Pod pojęciem \dcsstrong{kodu użytkownika} rozumiany jest
kod źródłowy pochodzący spoza biblioteki Bluepath -- będący częścią
aplikacji, która ją wykorzystuje. 
\end{defn}
Termin ten odnosi się najczęściej do fragmentów kodu, które są wykonywane
w~ramach rozproszonego wątku (def. \ref{def:background-Rozproszony-watek}). 




\subsection{Terminy technologiczne}

W pracy pojawia się szereg terminów technologicznych -- głównie akronimów
-- które, w celu uniknięcia nieporozumień, zdefiniowano poniżej.


\subsubsection*{XML}

\label{def:background-XML}XML (ang.~\dcsemph{Extensible Markup Language})
to uniwersalny język znaczników (zwanych tagami) służący do reprezentacji
danych.




\subsubsection*{XSD}

\label{def:background-XSD}XSD (ang.~\dcsemph{XML Schema Definition})
to język, który służy do opisywania w sposób formalny struktury elementów
dokumentu XML.




\subsubsection*{WCF}

\label{def:background-WCF}WCF (ang, \dcsemph{Windows Communication Foundation})
\cite{MSDN:WhatIsWindowsCommunicationFoundation} to środowisko przeznaczone
do implementacji usług sieciowych.




\subsubsection*{Zapora ogniowa}

\label{def:background-Firewall}Zapora ogniowa (ang.~\dcsemph{firewall})
to oprogramowanie filtrujące ruch sieciowy w oparciu o zestaw reguł.

Zapory ogniowe mają również możliwość modyfikacji pakietów, czego
przykładem może być mechanizm NAT (ang. \dcsemph{network address translation})
\cite{rfc1631,rfc3022} służący do zamiany adresu źródłowego, docelowego
oraz numerów portów w celu umożliwienia komunikacji z użyciem jednego
publicznego adresu IP wielu komputerom znajdującym się w sieci lokalnej
i korzystającym z prywatnych adresów IP.


