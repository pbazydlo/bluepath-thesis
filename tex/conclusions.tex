
\chapter{Podsumowanie}

Niniejszy rozdział stanowi podsumowanie pracy. Omówiono w nim przebieg
realizacji, napotkane w trakcie implementacji problemy oraz przedstawiono
potencjalne kierunki dalszego rozwoju projektu.


\section{Przebieg realizacji}

W trakcie realizacji tej pracy udało się zrealizować główny cel --
zaprojektowanie biblioteki wspomagającej użytkowników w implementacji
aplikacji rozproszonych (por. punkt \ref{sec:intro-cel-i-zakres}).
Wszystkie przyjęte założenia były weryfikowane poprzez implementację
prototypu opisaną w rozdziale \ref{chap:implementation}. W obecnym
momencie biblioteka posiada zależność jedynie od usługi Redis wykorzystywanej
do realizacji \dcsname{pamięci rozproszonej}. Należy jednak nadmienić,
że zależność tę można w łatwy sposób usunąć poprzez dostarczenie własnej
implementacji \dcsname{pamięci rozproszonej} co jest możliwe dzięki
modułowości biblioteki.

Biblioteka i założenia poczynione przy jej tworzeniu zostały dodatkowo
zweryfikowane poprzez implementację aplikacji opisanych w rozdziale
\ref{chap:applications}. Możliwość wykorzystania \dcsemph{debuggera}
w trakcie tworzenia tych aplikacji oraz testowania działania biblioteki
okazała się niezastąpiona i stanowiła duży atut.




\section{Napotkane problemy}

W trakcie realizacji pracy napotkano problemy, których opis i rozwiązania
opisano poniżej. Ich natura była bardzo różna, wynikały głównie z
błędów i ograniczeń w~wykorzystywanym kodzie pochodzącym od innych
twórców:
\begin{itemize}
\item serializacja wyjątków,
\item niezaimplementowane do końca wewnętrzne metody w Rhino DHT,
\item ograniczenie wydajności środowiska Redis w wyniku przenoszenia kodu
z systemu Linux do systemu Windows,
\item błędy składniowe w plikach XML Schema Definition pochodzących z bibliteki
OpenXES.
\end{itemize}

\subsection{Serializacja}

Wszystkie wyjątki w .NET Framework powinny być serializowalne. Okazuje
się jednak, że klasa bazowa \dcscode{Exception} zawiera wirtualne
pole \dcscode{Data} typu \dcscode{IDictionary}, które umożliwia
dodanie do kolekcji obiektu, który nie będzie serializowalny. Pomimo
tego, że każdy wstawiany obiekt jest sprawdzany pod tym kątem poprzez
sprawdzenie flagi \dcscode{Type.IsSerializable}, nie gwarantuje to,
że jego składowe będą serializowalne, a tym samym i cały wyjątek. 

Drugim problemem jest zachowanie klasy \dcscode{DataContractSerializer}.
Zakładając, że definicja wiadomości (\dcscode{DataContract}) zawiera
pole typu \dcscode{Exception}, w którym ma zostać przesłany wyjątek,
\dcscode{DataContractSerializer} wymaga dodania atrybutu \dcscode{KnownType}
dla każdego typu wyjątku wywiedzionego z typu \dcscode{Exception},
który ma zostać przesłany na zdalną stronę, również tego zawartego
w polu na wyjątek wewnętrzny (\dcscode{InnerException)}. Dodawanie
atrybutu \dcscode{KnownType} dla każdego typu wyjątku, który może
zostać wygenerowany przez kod użytkownika jest rozwiązaniem dalekim
od transparentnego, w związku z tym zdecydowano o przesyłaniu wyjątków
w formie tekstowej -- wywoływana jest na nich rekurencyjnie metoda
\dcscode{ToString}.


\subsection{Rhino DHT}

\label{par:problemy-rhino-dht}W punkcie \ref{sub:implementation-Pamiec-rozproszona-rozwazane-rozwiazania},
gdzie opisano wybór aplikacji do zrealizowania \dcsname{pamięci rozproszonej},
zasygnalizowano problem, który był powodem porzucenia biblioteki Rhino
DHT. Zgodnie z założeniami przedstawionymi w punkcie \ref{def:concept-Pamiec-rozproszona},
biblioteka powinna udostępniać mechanizmy atomowego porównania i zapisu
bądź atomowego odczytu i zapisu. Przykładowa implementacja \dcsname{zamka rozproszonego}
miała opierać się o mechanizm wersjonowania danych obecny w Rhino
DHT:
\begin{itemize}
\item pobranie aktualnej, najnowszej wersji pola reprezentującego wartość
zamka (wolny/zajęty przez wykonawcę o danym \dcscode{eid}) wraz z
numerem wersji \dcsemph{k},
\item w przypadku gdy zamek jest wolny -- zapis nowej wersji wartości pola
(własnego identyfikatora \dcscode{eid}) ze wskazaniem wersji \dcsemph{k}
jako rodzica,
\item odczyt wersji \dcsemph{k+1} i porównanie z własnym eid -- jeżeli
jest równe oznacza to, że zamek został pobrany, w przeciwnym wypadku
-- że został zajęty przez inny proces,
\item zwolnienie zamka odbywać się miała poprzez zapis odpowiedniej informacji
ze wskazaniem wersji \dcsemph{k+1} jako rodzica.
\end{itemize}
Widać zatem, że zapisy miały tworzyć strukturę drzewiastą, gdzie najdłuższa
ścieżka miała reprezentować sekwencję pobranych i zwolnionych zamków.
Niestety wewnętrznie nie wszystkie operacje zostały zaimplementowane,
co objawiało się wyjątkiem \dcscode{NotImplementedException} zgłaszanym
z wewnątrz biblioteki Rhino DHT.


\subsection{Redis uruchamiany w systemie Windows}

\label{par:problemy-Windows-Redis}Podczas pracy usługi Redis z dużym
obciążeniem pod kontrolą systemu Windows zaobserwowano znaczący spadek
wydajności. Wyjaśnienie problemów z działaniem znaleziono w dokumencie
\cite{MSOpenTech-Redis-Release-Notes}. 

W wersji dla systemu Linux wszystkie operacje wejścia-wyjścia korzystają
z deskryptorów plików, które nie są tak rozpowszechnione w API systemu
Windows. Wersja przeznaczona dla systemów Windows używa w związku
z tym symulowanego deskryptora plików. Zasadniczym problemem jest
jednak brak obecności wywołania systemowego \dcscode{fork} do tworzenia
procesów potomnych, która w systemie Windows została zasymulowana
poprzez umieszczenie sterty procesu Redis w \dcsemph{memory-mapped file},
czyli fragmencie pamięci wirtualnej, która jest dokładnym odwzorowaniem
pliku znajdującego się na dysku lub -- w ogólności -- obiektu, do
którego jesteśmy w stanie uzyskać deskryptor. Taki deskryptor przekazywany
jest następnie potomnym procesom. Warto zauważyć, że domyślnym zachowaniem
środowiska jest tworzenie współdzielonego pliku o rozmiarze odpowiadającym
rozmiarowi pamięci fizycznej komputera. 

Ostatecznie różnice, które ujawniły się pomiędzy wersją usługi Redis
przeznaczoną dla systemów Windows a pochodnych UNIXa, wpłynęły na
podjęcie decyzji o uruchomieniu go w systemie operacyjnym Ubuntu.


\subsection{Implementacja standardu OpenXES}

\label{sub:problemy-Implementacja-standardu-OpenXES-z-XSD}Podczas
próby implementacji standardu OpenXES opisanej w punkcie \ref{sec:implementation-Logowanie-zdarzen}
na podstawie plików XML Schema udostępnionych 28 marca 2014 r. wraz
z wersją 2.0 bibliteki napotkano problemy przy próbie automatycznego
wygenerowania klas. W~jednym z plików, \dcscode{xes.xsd}, znaleziony
został drobny błąd składniowy -- brakująca spacja, co powodowało,
że plik XML był niepoprawny składniowo i nieprzyjmowany przez narzędzie
\dcscode{xsd.exe}. Po jego poprawieniu ujawnił się kolejny problem
objawiający się wyjątkiem przepełnienia stosu w trakcie generowania
klas. Ten problem został rozwiązany przez objęcie wnętrza elementu
\dcscode{xs:complexType} odnoszącego się do typu \dcscode{AttributableType}
elementem \dcscode{xs:sequence}.


\section{Perspektywy dalszego rozwoju}

Przy projektowaniu biblioteki Bluepath został zdefiniowany szereg
założeń oraz podjęto szereg decyzji projektowych przedstawionych w
rozdziałach trzecim i czwartym. Poniżej wskazano potencjalne kierunki
rozwoju projektu -- rozluźniania pewnych założeń i rozbudowy funkcjonalności.


\subsection{Niezawodność przetwarzania}

\label{sub:conclusions-Obsluga-awarii}Aspektem, który nie został
poruszony w niniejszej pracy jest zapewnienie niezawodności przetwarzania
w przypadku awarii węzłów (por. punkt \ref{sec:concept-Obsluga-awarii}).
Poniżej zaproponowano sposoby obsługi awarii różnych komponentów --
mogą one dotyczyć węzłów danych, węzłów obliczeniowych a także \dcsname{usługi odnajdywania węzłów}.


\subsubsection*{Obsługa awarii węzłów danych}

Zapewnienie bezpieczeństwa węzłów danych leży po stronie wybranej
do ich zrealizowania aplikacji (por. punkt \ref{sec:concept-Rozproszona-pami=000119=000107-wsp=0000F3=000142dzielona}).
Wykorzystywany obecnie system Redis może zostać uruchomiony w trybie
z replikacją, w którym jeden z węzłów jest punktem dostępowym do zapisu
lub odczytu, a jego repliki udostępniają dane tylko do odczytu. Po
awarii węzła nadrzędnego jedna z replik staje się nowym punktem dostępowym
z możliwością zapisu.

Istnieje również wariant systemu -- Redis Cluster -- w którym dane
dzielone są pomiędzy węzły, udostępniający mechanizmy replikacji i
rekonfiguracji w przypadku awarii.


\subsubsection*{Obsługa awarii węzłów obliczeniowych}

W przypadku, gdy aplikacja nie posiada współdzielonego stanu, czyli
nie wykorzystuje rozproszonych struktur danych ani obiektów -- a więc
wywołania są idempotentne -- obsługa awarii węzłów obliczeniowych
mogłaby uwzględniać drzewiasty charakter takiego przetwarzania. \dcsname{Wątki rozproszone}
zainicjowane przez stracony węzeł musiałyby być wycofane, a węzeł
nadrzędny musiałby zlecić jeszcze raz wykonanie przerwanego wątku.
\dcsname{Wątki rozproszone}, które mogą być bezpiecznie uruchamiane
ponownie musiałyby być \dcsemph{explicite} oznaczane przez programistę
flagą (\dcscode{Resumable}) lub z wykorzystaniem dedykowanej metody
(\dcscode{AsResumable}). 


\subsubsection*{Obsługa awarii serwera usługi odnajdywania węzłów}

Wrażliwym punktem jest również scentralizowany serwer \dcsname{usługi odnajdywania węzłów}.
Warto jednak zauważyć, że prawdopodobieństwo awarii jednego konkretnego
węzła jest dużo niższe niż prawdopodobieństwo awarii dowolnego węzła
w klastrze. Argumentem, który przemawia za koniecznością zajęcia się
tym problemem jest kluczowe znaczenie serwera \dcsname{usługi odnajdywania węzłów}
do realizacji przetwarzania. Aby zapewnić niezawodność dostępu do
\dcsname{usługi odnajdywania węzłów} możnaby uruchomić jego replikę,
która przejęłaby zadania po awarii podstawowego serwera (podejście
to wykorzystywane jest np. w MapReduce, gdzie \dcsemph{name node}
jest replikowany). Warto również rozważyć implementację opartą o rozproszoną
tablcę haszową (ang.~\dcsemph{distributed hash table}, DHT) z replikacją.


\subsection{Rozbudowa funkcjonalności}

Ponieważ z przetwarzaniem rozproszonym wiąże się wiele problemów,
a przy tworzeniu biblioteki zauważano kolejne obszary w których możnaby
zmniejszyć nakład pracy użytkownika nie było możliwe zrealizowanie
wszystkich funkcji. Poniżej przedstawiono przykładowe obszary, w których
możnaby rozbudować funkcjonalność systemu:
\begin{itemize}
\item rozproszone struktury danych -- np. implementacja kolejki,
\item interfejs do komunikacji z systemem -- rozbudowa o metody dostępu
do pamięci lokalnej węzła (\dcscode{LocalStorage}), dodanie możliwości
raportowania zaawansowania przetwarzania w ramach wątku (\dcscode{ReportProgress})
w celu budowy dokładniejszych planistów,
\item DLINQ -- implementacja pozostałych metod obecnych w standardowym LINQ
i niedostępnych w obecnej implementacji,
\item wykorzystanie asynchronicznych wzorców z .NET Framework 4.5 (\dcscode{async},
\dcscode{await}) -- obecnie jedyną operacją asynchroniczną jest oczekiwanie
na zakończenie \dcsname{wątku rozproszonego} -- \dcscode{JoinAsync}.
\end{itemize}

\section{Zakończenie}

W niniejszej pracy przeanalizowano i porównano istniejące rozwiązania
wspomagające programistów w tworzeniu aplikacji rozproszonych oraz
zaprojektowano i zaimplementowano prototyp biblioteki, która pozwala
jej użytkownikom w prosty sposób zaimplementować program przetwarzający
dane w sposób rozproszony. Użytkownik pisząc program w sposób podobny
do tego, w jaki pisałby program równoległy, uzyskuje program wykorzystujący
moc obliczeniową wielu węzłów w klastrze. W takim przypadku wątki,
zamiast wykonywać się na wielu rdzeniach jednego procesora, wykonują
się na zdalnych maszynach. Prototyp biblioteki został przetestowany
zarówno pod względem jakościowym jak i wydajnościowym. Przedstawiono
również przykładowe zastosowania biblioteki. Biorąc powyższe pod uwagę
można uznać, że wszystkie założone cele pracy zostały zrealizowane.
