
\chapter{Wstęp}
\begin{quote}
\dcsemph{Programming is intrinsically very difficult.}

\begin{flushright}
--- E. Dijkstra
\par\end{flushright}
\end{quote}
Tworzenie aplikacji działających w klastrze jest złożone i wiąże się
z nim wiele problemów takich jak wykrycie zakończenia czy dystrybucja
danych pomiędzy maszynami. Dużo prostsze jest tworzenie aplikacji
wykorzystujących wiele procesorów i rdzeni pojedynczej maszyny. W
niniejszej pracy zaproponowano rozwiązanie czerpiące z obu podejść,
pozwalające zastosować mechanizmy znane z programowania równoległego
przy tworzeniu aplikacji rozproszonych.

W tym rozdziale przedstawione zostaną założony cel oraz zakres pracy,
jej struktura, motywacja oraz podział prac pomiędzy autorów.


\section{Cel i zakres pracy}

\label{sec:intro-cel-i-zakres}Celem pracy było zaprojektowanie biblioteki,
która, bez skomplikowanej konfiguracji czy instalowania wielu składników,
pozwoli jej użytkownikom, programistom, w prosty sposób zaimplementować
program przetwarzający dane w sposób rozproszony. Programista pisząc
program w sposób podobny do tego w jaki pisałby program równoległy,
powinien uzyskać program wykorzystujący moc obliczeniową wielu węzłów
w klastrze. W takim przypadku wątki, zamiast wykonywać się na wielu
rdzeniach jednego procesora, mogą wykonywać się na zdalnych maszynach.


\section{Struktura pracy}

W niniejszej pracy omówiono koncepcję oraz implementację biblioteki
o kodowej nazwie Bluepath. Bieżący rozdział opisuje w dalszej części
motywacje oraz wkład członków zespołu. Rozdział drugi stanowi wstęp
teoretyczny obejmujący definicje oraz przegląd istniejących rozwiązań
-- środowisk przetwarzania rozproszonego. W~rozdziale trzecim została
szczegółowo przedstawiona koncepcja i projekt systemu, a~rozdział
czwarty opisuje implementacje poszczególnych komponentów. Przykładowe
zastosowania biblioteki przy implementowaniu aplikacji obliczeniowych
przedstawia rozdział piąty, a wyniki testów jakościowych, wydajnościowych
i ich analizę -- rozdział szósty. W rodziale siódmym podsumowano przebieg
realizacji, opisano napotkane problemy a także zaproponowano obszary,
w których możnaby kontynuować prace nad usprawnieniem biblioteki.


\section{Motywacja}

W czasie, kiedy rozpoczynano pracę nie było dostępne popularne, aktywnie
rozwijane i wspierane komercyjnie środowisko do przetwarzania rozproszonego
dla .NET Framework (por. punkt \ref{sec:background-Istniej=000105ce-rozwi=000105zania}).
Tego typu środowisko pozwoliłoby użytkownikom uniknąć konieczności
samodzielnego rozwiązywania wielu problemów przetwarzania rozproszonego
oraz zaoszczędzić czas poprzez wykorzystanie dostarczonych mechanizmów.


\section{Udział w pracy poszczególnych członków zespołu}

Niniejsza praca została stworzona przez dwie osoby. Podział prac został
przedstawiony poniżej.


\paragraph{Piotr Bazydło}
\begin{itemize}
\item koncepcja rozwiązania,
\item przegląd i analiza mechanizmów pamięci współdzielonej,
\item przykładowe aplikacje -- DLINQ, system uzupełniania wyrazów,
\item implementacja części testów jakościowych i wydajnościowych,
\item przygotowanie środowiska do przeprowadzenia testów wydajnościowych,
\item implementacja zamków rozproszonych,
\item implementacja planisty szeregującego zadania w oparciu o obciążenie
węzłów,
\item implementacja planisty szeregującego zadania wykorzystującego algorytm
cykliczny,
\item implementacja pamięci rozproszonej na bazie istniejącego rozwiązania,
\item implementacja rozproszonych struktur danych i obiektów.
\end{itemize}

\paragraph{Zachariusz Karwacki}
\begin{itemize}
\item przegląd istniejących rozwiązań,
\item koncepcja rozwiązania,
\item implementacja logiki wątku rozproszonego,
\item implementacja zdalnego wykonawcy,
\item implementacja lokalnego wykonawcy,
\item implementacja mechanizmów logowania zdarzeń,
\item implementacja usługi odnajdywania węzłów,
\item implementacja zarządcy połączeń,
\item implementacja interfejsu do komunikacji z systemem,
\item implementacja części testów jakościowych i wydajnościowych,
\item przykładowe aplikacje -- MapReduce, obliczanie liczby PI,
\item stworzenie skryptów dystrybuujących aplikację w klastrze.\end{itemize}

